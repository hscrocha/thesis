\documentclass{article}
\usepackage{arxiv}
\usepackage[utf8]{inputenc}
\usepackage[T1]{fontenc}    % use 8-bit T1 fonts
\usepackage{hyperref}       % hyperlinks
\usepackage{url}            % simple URL typesetting
\usepackage{booktabs}       % professional-quality tables
\usepackage{amsfonts}       % blackboard math symbols
\usepackage{amsmath}
\usepackage{nicefrac}       % compact symbols for 1/2, etc.
\usepackage{microtype}      % microtypography
\usepackage{lipsum}
\usepackage{graphicx}
\usepackage{float}
\usepackage{subcaption}
\usepackage[ruled,vlined]{algorithm2e}
\usepackage{cleveref}

\usepackage{todonotes}
\newcommand{\comment}[1]{\todo[color=orange!40, inline]{\footnotesize{#1}}}
\newcommand{\kleidi}[1]{\todo[color=green!40, inline]{\footnotesize{Kleidi: #1}}}
\newcommand{\henrique}[1]{\todo[color=blue!40, inline]{\footnotesize{Henrique: #1}}}
\newcommand{\yves}[1]{\todo[color=yellow!40, inline]{\footnotesize{Yves: #1}}}
\newcommand{\serge}[1]{\todo[color=purple!40, inline]{\footnotesize{Serge: #1}}}


\title{Ranking the output of static analysis}
\author{Kleidi Ismailaj \\ \\ \href{mailto:kleidi.ismailaj@student.uantwerpen.be}
{\textit{kleidi.ismailaj@student.uantwerpen.be}}}
\date{\today}

\begin{document}

\maketitle
\newpage
\tableofcontents
\newpage

\section*{Abstract}
\addcontentsline{toc}{section}{Abstract}
\henrique{It is very rare to have a reference in an Abstract. We should be careful and verify if it is really necessary.}
\kleidi{It is a statement made by most of the papers regarding the subject. I could remove the citations.}


Previous research has estimated the number of false positive static analysis alerts to be as high as 70\%. Research has also shown that developers loose trust and ignore such tools when the number of false positive (or unimportant) alerts is high. Given that static analysis is still helpful in promoting cleaner code, different aspects from the history of the codebase can be exploited to improve the relevance of the output. If the output is matched to what the developer finds important or what has been proven to be helpful in the past, this will lead to a better acceptance and wider usage of such tools.

% I try to have four sentences in my abstract. The first states the problem. The second states why the problem is a problem. The third is my startling sentence. The fourth states the implication of my startling sentence. An abstract for this paper done in this style would be:

% The rejection rate for OOPSLA papers in near 90\%. Most papers are rejected not because of a lack of good ideas, but because they are poorly structured. Following four simple steps in writing a paper will dramatically increase your chances of acceptance. If everyone followed these steps, the amount of communication in the object community would increase, improving the rate of progress. 

% !TEX root = main.tex

\section{Introduction}\label{sec:introduction}
% This is a general introduction to what the thesis is all about -- it is not just a description of the contents of each section. Briefly summarize the question (you will be stating the question in detail later), some of the reasons why it is a worthwhile question, and perhaps give an overview of your main results. This is a birds-eye view of the answers to the main questions answered in the thesis (see above). 

% +Positive aspects of SA and usages (google, fb etc...), their need (huge amounts of code), etc...\\

\henrique{Is the SA used for maintenance, development, or both? If it is maintenance, we should shift the initial paragraph wih a little motivation on maintenance. If it for both, we should focus on development but also mention maintenance activities.}
\kleidi{I know that OMP wants to use it in a CI environment, so in development}
\henrique{I would argue that CI is both development and maintenance.}
\kleidi{Basically when devs push a change, th CI will analyze it and if there are warnings it will ask the devs to pay attention to them (its very vague, and i don't have much info but i think its more development)}

%The amount of code that is being produced is increasing with time. \henrique{It is important, to put a reference when writing such strong statement. Although, I think there are better ways to start the intro.} That brings multiple challenges in the software engineering landscape, one of which is assuring adequate code quality. In this context, automatic approaches such as Static Analysis (SA) can be really useful and time saving in detecting and preventing potential bugs. \henrique{The last two sentences are confusing, first you say it is important to assure code quality. And next you talk about bugs. Those are different concepts, you can have a code with bad quality and low bugs and vice-versa. From what I understand, your goal is more related to bug prevention. Quality metrics for source cde are usually based on coupling and cohesion, and you do not measure that.} 
%Its usefulness has been acknowledged also by big tech companies, such as Google with its Tricorder architecture~\cite{sa_google} or Facebook with its Infer static analyzer~\cite{infer}.
%Another reason to adopt SA techniques, is that the cost of repairing a defect is much lower when that defect is found early in the development cycle (generally accepted principle). \henrique{Even if it is a generally acepted principle, as academic researchers we should provide a reference to it. Here it is a journal paper that shows a figure on the costs measured by the IBM System Science Institute~\cite{dawson2010}.}
%
%\henrique{Bellow is my atempt to rewrite the first paragraph. You can use whichever you prefer, or mix and mash what you want. I think there are 3 different topics above that do not match all that well. Therefore, I split it into 3 paragraphs}

Software development and maintenance is a complex activity~\cite{clarke2016}. Because of this complexity, there are many challenges in the software engineering landscape, when developing software. In this environment, automated tools and techniques matter to help the development scale with consistency~\cite{winters2020}. 

Static Analysis (SA) is an automated technique that can be useful for the development process. For example, it is possible to use SA to detect potential bugs in the source code. It is a generally accepted principle that resolving issues in earlier stages of the development cycle is less costly~\cite{dawson2010}. Therefore, the bugs detected by SA may reduce the maintenance and development costs.

Popular companies acknowledge the usefulness of Static Analysis techniques by adopting them in their process. For instance, Google with its Tricorder architecture~\cite{sa_google} or Facebook with its Infer static analyzer~\cite{infer}.

%\henrique{** End of the first paragraph rewrite suggestion **}


Even though SA is useful and employed by real companies, there is still much room for improvement. Since the software under analysis is not executed, SA tools must infer what the actual program behavior will be. Therefore, SA tools are bound to make misclassifications or raise false alarms.

Given the tendency of SA tools to over-estimate possible faulty program behaviours, there is a need to fine-tune and improve the alarms and warnings raised by these tools. There are two ways that we can improve SA tools: (i) increase the precision of the analysis which usually decreases its recall and the overall number of raised alarms; or (ii) by post-processing the alarms after they are generated, according to specific criteria more appropriate for the codebase. We opted to focus on the second option, post-processing of alarms. 
%%
The classic approach most tools use for prioritizing and filtering results is to classify the results based on severity levels. This approach is simple but oblivious to the actual analyzed code or the location and frequency of a given issue. Furthermore, it has been shown that if developers lose trust in the tool, they then tend to ignore the output of it altogether \cite{sa_google}.

Different approaches have been proposed to improve the output of SA tools. Optimally, the initial warnings report should be those most likely to be real errors. Alerts can be strategically prioritized for examination, by tracking and analyzing them through a series of software versions, we can automatically determine which SA rules are more important and which parts of the software are more problematic. 
% Furthermore, alerts can be grouped/clustered in basis of their similarity, so that users can check only a few and then can easily determine if the rest of the errors are also worth inspecting.\\
Moreover, an understanding of how developers react to these alerts can help improve the usability of these tools. Alerts can be divided into two categories: (i) actionable alerts (AA) which the programmer would act on to resolve; and (ii) unactionable alerts (UA) which the programmer would not act on~\cite{comparative_heckman, actionable_sa}. An unactionable alert may be of trivial concern to fix, less likely to manifest at runtime, or incorrectly identified due to the limitations of the tool. Thus, we want to prioritize the actionable alerts and hide from developers the unactionable ones.

Another approach is to prioritize alerts that, in the past, lead to the discovery of bugs~\cite{which_warnings,automatic_training_set}. By tracking back the bugs up to a past version, we can collect code lines that changed during bug fixes. Subsequently, we can pinpoint which alerts warned about those specific parts of code and prioritize accordingly.

For a relatively large project, the number of SA alerts can be prohibitive, which is one of the reasons developers avoid these tools~\cite{why_dont_use}. The ultimate goal is not to analyze all alerts, but to maximize the time-cost spent on them. Ranking schemes do not reduce alert investigation burdens if the aim is to check them all. Instead, they solve the problem by showing alerts that are most likely to be useful, so that developers can spend time by inspecting the most important ones.

Different approaches have been proposed in the literature but few have tried to do a comparison in terms of the utility of each method. Among those few, the comparison is mainly done between open-source Java software where a good amount of data can be extracted. In our case, we assess the utility of different approaches inside an industrial C++ codebase, where static analysis has been abandoned due to poor performance. Also, it is interesting to explore if these methods can be combined to achieve better results. Different techniques can be better suited to different types of alerts or can compensate for each others weaknesses if combined.

% (depending also on the information that can be extracted from the source code).

% -what is this student's research question?
% -is it a good question? (has it been answered before? is it a useful question to work on?) 

%Given an industrial code base, where SA tools have been abandoned because of the large amount of false positives, the goal of this thesis is to test the feasibility of successfully applying these ranking approaches even in a context with limited amount of (or noisy) data.

%\henrique{This my re-write on the previous paragraph. Again, use whichever you prefer or mix then both.}
In this thesis, our goal is to verify if it is possible to apply ranking approaches to static analysis alerts in an industrial code base with a limited amount of noisy data. We like to highlight that SA tools were initially employed and later abandoned by this company due to a large number of false positives. We claim that a ranking mechanism fined tuned to this company's code base can achieve good results and be useful to developers.

% The research questions can be formulated as follows:
% \begin{itemize}
%     \item R0: Can we apply SA ranking techniques in an industrial environment with limited amount of data (abandoned because of high false positives)?
%     \item R1: Given a highly imbalances dataset, what are the best performing dataset balancing techniques?
%     \item R2: Can we process the collected data to remove faulty examples (CLNI?) (project specific bugs?) (fix the data not the algorithms)
%     \item R3: Can we combine SA ranking techniques to achieve better results?
% \end{itemize}

% In this thesis, a comparison of different SA processing approaches is performed in an industrial codebase. Also, the utility of combining different methods is explored.

The rest of this thesis will be structured in the following sections... (\textbf{TODO})

% \textbf{TO DO: duality, actionable alert and bug predicition alerts}


% Arguments/information gotten from papers that can be used for the thesis introduction.
% \begin{itemize}
%     \item Why the false positives?
%     \begin{itemize}
%         \item Inevitable mistakes (SA has to make a trade-off with efficiency).
%         \item Static analyzers commonly use heuristics or approximations to determine properties for a given code construct, which frequently induces them to make inaccurate assumptions regarding the behavior of the program under evaluation.
%         \item A tool may deliberately introduce approximations for scalability or speed or to check richer properties than is generally possible, potentially resulting in an outbreak of errors. By ranking the results the invalid errors can be relegated below true errors.
%         \item Because the software under analysis is not executed, static analysis tools must speculate on what the actual program behavior will be. They often over-estimate possible program behaviors, leading to spurious warnings (“false positives”) that do not correspond to true defects.
%     \end{itemize}

%     \item Why its important to rank warnings (show the most relevant subset) or reduce false positive?
%     \begin{itemize}
%         \item The classic approach most automated code inspection tools use for prioritizing and filtering results is to classify the results based on severity levels. Such levels are (statically) associated with the type of defects detected; they are oblivious of the actual code that is being analyzed and of the location or frequency of a given defect.
%         \item Empirically, all tools that effectively find errors have false positive rates that can easily reach 30–100\%.
%         \item In general, there are roughly 40 warnings for every thousand lines of code [11]. This overload of warnings is a prime reason for developers to avoid using ASATs [12]. (\cite{analysis_sa_usage})
%         \item False reports can easily render tools useless by hiding real errors amidst the false, and by potentially causing the tool to be discarded as irrelevant.
%         \item Warnings are not always acted on by developers even if they reveal true defects. Reasons for defects being ignored include warnings implicating obsolete code, “trivial” defects with no impact on the user, and real defects requiring significant effort to fix with little perceived benefit.
%         \item The effort required to manually audit all alerts and repair all confirmed code flaws is often too much for a project’s budget and schedule.
%     \end{itemize}

%     \item How to improve?
%     \begin{itemize}
%         \item An understanding of how developers react on the alerts detected by SA tools can help improve the utility of these tools and determine future research directions.
%         \item The initial few error reports should be those most likely to be real errors so that the user can easily see if the rest of the errors are worth inspecting (otherwise users tend to discard the tool).
%         \item \textbf{Strategically prioritizing} alerts for examination.
%         \item Results indicate that using several ASATs has benefits over using just a single ASAT. (\cite{analysis_sa_usage})
%         \item Tracking warnings through a series of software versions reveals where potential defects are commonly introduced and addressed, and how long they persist; thus exposing interesting trends and patterns. This helps in determining which SA rules are important for a software system and helps select a minimum set of rules that must be enabled.
%     \end{itemize}

%     \item We use the term actionable alert (AA) to define a SA alert that the programmer would act on to resolve and unactionable alert (UA) to define a SA alert that the programmer would not act on to resolve.
    
%     \item Only show \textbf{actionable alerts}.
%     An unactionable alert may be one of the following: 1) a trivial concern to fix, 2) less likely to manifest in runtime environment; or 3) incorrectly identified due to the limitations of the tool.
    
    
%     % \item It is generally accepted that the cost of repairing a defect is much lower when that defect is found early in the development cycle.
    
%     % \item The above problem of alarms and associated cost can be addressed either by improving precision of the analysis or by postprocessing the alarms effectively after they are generated.
    
%     % \item Note that the above described handling of alarms does not consider reducing the number of alarms \textbf{by making underlined static analysis more precise}. That is, it excludes the option of improving precision of analyses, like value analysis and pointer analysis, implemented in the static analysis tools.
    
%     % \item However, even when tools focus on uncovering the same defect type, the variance in defects found is still very large [6],[13]–[15]. These results indicate that using several ASATs has benefits over using just a single ASAT. (\cite{analysis_sa_usage})
    
%     \item ranking schemes do not reduce alarm-investigation burdens - alleviate the false alarm problem by showing alarms that are most likely to be real errors over those that are least likely. However, the number of alarms to investigate is not reduced with ranking.
    
% \end{itemize}


% !TEX root = main.tex

\section{Problem statement}\label{sec:problem-statement}
%\henrique{To me, a problem statement section should appear as earlier as possible in the thesis. It can even be a subsection of the introduction. }
% Research Question or Problem Statement

% Engineering theses tend to refer to a "problem" to be solved where other disciplines talk in terms of a "question" to be answered. In either case, this section has three main parts:

% 1. a concise statement of the question that your thesis tackles
% 2. justification, by direct reference to the literature review, that your question is previously unanswered
% 3. discussion of why it is worthwhile to answer this question.

% Item 2 above is where you analyze the information which you presented in Section 3. For example, maybe your problem is to "develop a Zylon algorithm capable of handling very large scale problems in reasonable time" (you would further describe what you mean by "large scale" and "reasonable time" in the problem statement). Now in your analysis of the state of the art you would show how each class of current approaches fails (i.e. can handle only small problems, or takes too much time). In the last part of this section you would explain why having a large-scale fast Zylon algorithm is useful; e.g., by describing applications where it can be used.

% Since this is one of the sections that the readers are definitely looking for, highlight it by using the word "problem" or "question" in the title: e.g. "Research Question" or "Problem Statement", or maybe something more specific such as "The Large-Scale Zylon Algorithm Problem." 

Improving the warnings provided by static analysis techniques dependents on a particular codebase and the people who produce it. Organizations can have different priorities and expectations on the static analysis alert report. Also, developers might be more interested in a particular subset of alerts, or the context on which the alerts appear.

%\henrique{There is an issue with the way you are writing this section. This is the problem statement section, but you barely explained the problem. Paragraph 2 is already talking about your goal. It would be better if you create another paragraph after Paragraph 1 with more detail on the problem. For example describing how SA were initially used by the company and later abandoned. That is a problem, and very important as motivation for your thesis.}
%\kleidi{The only info I had is that SA use was discontinued due to high amount of FP, unfortunately don't have much more to add.}
%\henrique{That sentence alone is already a nice set-up to a real problem affecting a real company.}
%\kleidi{added it}

OMP regards static analysis as a useful approach in improving software quality and preventing bugs early in the development cycle. Though, the usage of such tools was stopped because of the high amount of false positives.

Starting from the original alert set produced by a SA tool, automatic processing techniques can be applied to rank the alerts in such a way that the utility of inspecting the ranked alerts is higher than the default order (there are more actionable alerts in the post-processed output). Given an industrial codebase, the goal is to explore if these automatic techniques are useful, which produces the best results, and if an ensemble technique provides extra benefits. The starting point is the version control history of the project. By extracting information about the past versions, an attempt can be made to learn which SA alerts are more important and can be prioritized in the future. In contrast to open source project where you can test the approaches only on those project that have a sufficient amount of data, in an industrial codebase you have to make the most of the data you can extract. This thesis also examines which ML techniques can be used to deal with imbalanced or noisy data.
%\henrique{I am getting confused here. I thought your approach was focused on post-processing SA alert by ranking them. But by reading the last paragraph I got a completely different idea. It seemed that your goal is to find the SA technique which gives better results (that is different than ranking), and combine them to find better results (again, no explicit mention of ranking).}
%\kleidi{Added a sentence in the begin to set the context better.}
%\henrique{Much better indeed. The sentence you added provided a more clear context.}


Two main approaches are explored: detecting actionable alerts (alerts deemed useful by the developers) and alerts that aid in detecting bugs. These approaches are complementary because the sets of alerts are not necessarily equal, thus they are a good candidate for a combination.

%\henrique{Research Questions are usually abreviated as RQ and not just R. Moreover, we start counting at 1.}
The research questions can be formulated as follows:
\begin{itemize}
    \item RQ\#1: Can we apply SA ranking techniques in an industrial environment with limited amount of data (abandoned because of high false positives)?
    \item RQ\#2: Can we combine SA ranking techniques to achieve better results?
    \item RQ\#3: Do pre-processing techniques provide a significant performance benefit?
\end{itemize}

%\henrique{Another thing that I noticed. You detail the RQ here, but they do not appear ever again in the thesis. When we create RQs, we say during the experimental setup (you Section 5) or evaluation (Section 6) which experiment is trying to assess/prove/verify/answer which RQ. Just putting RQs here without matching them later in your experiments/evaluation is not good.}
%\kleidi{Will add the section later in the experiments/conclusion}

This research is important because it quantitatively examines if the version history of a project combined with machine learning techniques can be used to effectively improve the output of SA tools. By doing so, we can examine the real utility of this approach in practice and identify which techniques are more effective. It is also relevant to see if these approaches produce meaningful results in the case of limited amount of data. We can also observe the impact in performance by testing different ML techniques to reduce noise and balance the dataset (under and oversampling). 

Based on the literature review, few papers make direct comparisons between different methods on a common experiment baseline \cite{comparative_heckman, compare_framework}. Also, they mainly focus on open source Java systems with an adequate amount of data. Kim et al. \cite{noise_defect} research the impact of noisy data and propose a solution . Regarding ensemble techniques, there have been approaches where multiple SA tools are combined, or where each alert types is handled by its own classifier. In contrast, we focus on C++ code and compare different preprocessing techniques. Also, we try an ensemble approach with two different methods of ranking SA alerts.


% !TEX root = main.tex

\section{Background}\label{sec:background}

\henrique{A Background section is not an Appendix. Specially for a thesis such as this, the main concepts you want to describe are important for a reader to better understand the techniques used.}

% What goes in the appendices? Any material which impedes the smooth development of your presentation, but which is important to justify the results of a thesis. Generally it is material that is of too nitty-gritty a level of detail for inclusion in the main body of the thesis, but which should be available for perusal by the examiners to convince them sufficiently. Examples include program listings, immense tables of data, lengthy mathematical proofs or derivations, etc. 

Possible concepts to explain/introduce in the thesis background section:
\begin{itemize}
    \item Bayesian networks
    \item ML algorithms
    % \item Logistic regression
    % \item Principal component analysis
    % \item Correlation methods
    \item Precision, recall, auc \henrique{We usually describe Precision, Recall and other metrics used in the Evaluation as a subsection inside the evaluation.}
    % \item Deep learning
    % \item Clang AST (if used)
\end{itemize}


% !TEX root = main.tex
\section{Literature review}\label{sec:literature}

This section will consist of research papers focused on these main topics: 
\begin{itemize}
    \item \textit{Ranking static analysis alerts}: some of the most known (cited) approaches to rank alerts.
    \begin{itemize}
        \item Using a single SA tool (\cref{sec:single_tool}).
        \item Combining multiple tools  (\cref{sec:multiple_tools}).
        \item Looking at survey papers that describe the state of the research and state of the art techniques (\cref{sec:lit_review}).
        \item Comparative studies evaluating different methods (\cref{sec:comparative_studies}).
    \end{itemize}
    \item \textit{Bug prediction}: Rank the alerts in problematic parts of code higher (\cref{sec:bug_prediction}).
    \item Information on \textit{real world usage of SA tools}, problems and suggested solutions (\cref{sec:tools_practice})
\end{itemize}

 \subsection{Dealing with False Positives}

 \subsubsection{Single Tool}
 \label{sec:single_tool}

 \label{lit:zranking}
 Kremenek and Engler \cite{z-ranking} introduce \textit{Z-ranking}, a statistical model to rank the error reports of SA tools. They make a distinction between successful and failed checks (those that satisfy a checked property and those that violate it). The underlying observation is that the most reliable error reports are those that generated few failed checks and many successful checks, since the actual amount of bugs in code is relatively small. An explosion of failed checks is a likely indicator that something is going wrong with the analysis. Reports are sorted based on the calculated \textit{z-test} statistic (based on the relative frequency of successful and failed checks).

 The problem can be formally defined as a classification task. Let P be the population of all reports, both successful checks and failed checks, emitted by a program checker analysis tool. \textit{P} consists of two subpopulations: \textit{S}, the subpopulation of successful checks and \textit{E}, the subpopulation of failed checks (or error reports). The set of error reports \textit{E} can be further broken down into two subpopulations: \textit{B}, the population of true errors or bugs and \textit{F}, the population of false positives. The classification problem can then be restated as follows: given an error report $x \in E$, decide which of the two populations \textit{B} and \textit{F} it belongs to. That is based on the fact that \textit{B} and \textit{F} have different statistical characteristics. 

 Given a grouping operator \textit{G} that groups successful and failed checks together, we calculate the proportion of failed checks $G.\rho = \frac{G.successful}{G.failed}$. Populations are ranked both by the $\rho$ value and by the degree of confidence in its estimation. By treating these checks inside the groups as a sequence of binary trials (coin tosses). The probability $p_i$ of success will have to be approximated using the standard error. By using the \textit{z-test} statistic, which measures how far an observed value is from the real population, a value can be specified that produces a large positive \textit{z-score} when there are few errors and many successes, and a large negative \textit{z-score} when there are few successes and many errors.

 Given an estimated $p_i$ and a calculated \textit{SE}, we can chose $p_0$ to produce the effect mentioned above:
 $z=\frac{observed-expected}{SE}=\frac{p_i-p_0}{SE}=\frac{p_i-p_0}{\sqrt{\frac{p_0(1-p_0)}{n}}}$. The average population success rate can be chosen as a starting point for the value of $p_0$.

 According to their tests, \textit{Z-ranking} performed better than randomized ranking 98.5\% of the time. Moreover, within the first 10\% of reports inspected, \textit{Z-ranking} found 3-7 times more real bugs on average than found by randomized ranking.\\

 \begin{figure}[H]
     \begin{subfigure}{1\textwidth}
         \centering
         \includegraphics[scale=0.7]{./src/z_ranking_result_linux_intra.png}
         \caption{Intra-procedural check on Linux spin (lock)}
     \end{subfigure}\\
     \begin{subfigure}{1\textwidth}
         \centering
         \includegraphics[scale=0.7]{./src/z_ranking_result_linux_inter.png}
         \caption{Inter-procedural check on Linux (free calls)}
     \end{subfigure}
 \end{figure}


 \label{lit:fbrank}
 Kremenek et al. \cite{correlation_exploitation} introduce \textit{Feedback-Rank}, a dynamic ranking scheme that adapts as reports are inspected. By analyzing historical data, they observed that both bugs and false positives cluster by code locality. They present a probabilistic technique that exploits this correlation and also incorporates user feedback by reordering reports after each inspection. Since reports are correlated within a population (cluster), inspecting one of them yields information about the others. The ranking works by using a \textit{Bayesian Network} and exploiting two features, the number of populations (error messages grouped together) and the strength of correlation in each population. Furthermore, the strategy also continues improving with time, by taking into account the history of inspections.

 An intuitive explanation for the reason that reports cluster is given for both real and false positives. Regarding true positives, when developers do not know a rule, they will repeatedly violate it, so errors of the same type will correlate together. As for the false positives, there are three main causes: analysis mistakes of the tool (explosion of errors), rare coding idioms used by developers (which trigger the tools), incomplete rule specifications (a rule holds in most cases, but can be safely violated in others).

 To cluster the reports, code locality in different granularities is chosen: function, file and directory level. From the results in \cref{fr:lshape} it can be seen that very few populations contain a mix of bugs and false positives. \textit{Applicability} is defined as the ratio of non singleton clusters (which are bad for online ranking) to the total amount of clusters. The coarser the granularity the greater the applicability but also the smaller the correlation. \textit{Skew} is defined as the ratio of homogeneous clusters (all bugs or all false positives) to the total number of clusters. In this case, the more refined the granularity (function level), the higher the skew. Thus, a trade-off needs to be made between applicability and skew.

 To apply the algorithm a model is needed that produces the correlations among the reports. The reports are divided into two major regions, one that contains mostly true positive ($g_B$), and one that contains mostly false positives ($g_{FP}$) (see \cref{fr:partition}). A \textit{Bayesian Network} is used to calculate the probabilities of a cluster belonging to a certain region (regions are different for different granularities). The initial configuration can either be chosen by the user or learned from historical data. A simple model can be seen on \cref{fr:bayes}, where 3 reports depend on the probabilities of the parent function, file and directory clusters they belong to. Influence though, flows across both directions: if we inspect a report and know its value, the probabilities of the parents are re-calculated. Gives a training set, the conditional probability distributions of the network (along with the probabilities for the regions) can be learned using \textit{Expectation Maximization}. \textit{Belief Propagation} is used to update probabilities after each inspection and \textit{Information Gain} is used as a secondary factor to rank the reports. 

 Feedback-Rank represents a complementary approach to static ranking schemes (it can be combined with Z-Ranking for example) and can be trained with other forms of correlation instead of code locality. 

 According to their tests, \textit{Feedback-Rank} performed 2-8 times better than randomized ranking.\\

 \begin{figure}[H]
     \begin{subfigure}{1\textwidth}
         \centering
         \includegraphics[scale=0.4]{./src/feedback_rank_l_shape.png}
         \caption{"L" shape of clustering}\label{fr:lshape}
     \end{subfigure}\\
     \begin{subfigure}{.5\textwidth}
         \centering
         \includegraphics[scale=0.6]{./src/feedback_rank_partition.png}
         \caption{Populations divided into regions, mostly true or false positives}\label{fr:partition}
     \end{subfigure}%
     \begin{subfigure}{.7\textwidth}
         \centering
         \includegraphics[scale=0.4]{./src/feedback_rank_bayes.png}
         \caption{Sample Bayesian network with 3 reports}\label{fr:bayes}
     \end{subfigure}
 \end{figure}


 Boogerd and Moonen \cite{static_profiling} present a technique, \textit{ELAN}, that prioritizes SA warnings by using the (predicted) likelihood that the execution reaches the location for which the warnings are reported. The execution likelihood is defined as the probability that a program point will be executed at least one in an arbitrary program run and is calculated statically. This computation is demand-driven, thus it is only performed for the locations associated with warning reports.

 The workflow (\cref{elan:workflow}) consists of normalizing the results of SA tools (to a specific format), creating system dependency graphs, calculating for every warning the likelihood of execution, ordering the results using the execution probability and possibly other external techniques (like Z-Ranking). 

 Likelihood analysis is based on system dependency graphs, which tie all program dependency graphs (function level) together by modelling the inter-procedural control dependencies. Their approach only considers control flow and ignore dataflow information. In order to avoid traversing all the SDG, \textit{program slicing} is used on control points. Other than the basic algorithm, they also introduce branch prediction heuristics, which do not excessively impact performance.

 Experiments show that predicted execution likelihoods correlate with data extracted from dynamic profiling. One problem though, is that when for example 30\% of all code is always executed, then the ranking of those warnings that belong to that piece of code, cannot be distinguished.

 \begin{figure}[H]
     \centering
     \includegraphics[scale=0.4]{./src/elan_workflow.png}
     \caption{Workflow for the ELAN tool}\label{elan:workflow}
 \end{figure}


 Kim and Ernst \cite{which_warnings} propose a history-based warning prioritization (HWP) algorithm which works by mining fix-changes in the VCS. It is based in the intuition that if a warning is removed by a fix, then probably that warning was important. On the other hand, if a warning instance is not removed for a long time, then warnings of
 that category may be neglectable, since the problem was not noticed or was not considered worth fixing. They measured the tool warning prioritization (TWP) on three different systems and found a precision of 3\%, 12\% and 8\%.  

 They set a weight to each warning category to represents its importance. The weight will be proportional to the number of warnings eliminated by changes (where fix-changes have the biggest weight, \cref{which_warnings:weights}). Selecting the top weighted warnings improves precision up to 17\%, 25\%, and 67\% respectively. Precision is calculated as $precision = \frac{number\:of \:warnings\:on\:bug\:related\:lines}{total\:number\:of\:warnings}$.
 By looking at the fix-changes and corresponding affected lines, by starting at the last revision, they can mark the bug-related lines, up to the first revision when they appeared (\cref{which_warnings:marking}). Ranking is category-based, so only the categories of warnings are considered and there is no distinction between the warnings inside each category. The algorithm works well if the categories are fine grained and internally homogeneous.

 They measure precision by training the weights in the first half of the version history, and testing them on the other half. The HWP outperforms TWP for all three systems (\cref{which_warnings:results}).

 \begin{figure}[H]
     \begin{subfigure}{0.6\textwidth}
         \begin{subfigure}{.5\textwidth}
             \centering
             \includegraphics[scale=0.45]{./src/which_warnings_weights.png}
             \caption{Precision results at line-level}\label{which_warnings:weights}
         \end{subfigure}\\
         \begin{subfigure}{.5\textwidth}
         \centering
         \includegraphics[scale=0.35]{./src/which_warnings_marking.png}
         \caption{Line marking approach}\label{which_warnings:marking}
         \end{subfigure}
     \end{subfigure}%
     \begin{subfigure}{0.4\textwidth}
         \centering
         \includegraphics[scale=0.5]{./src/which_warnings_results.png}
         \caption{Line marking approach}\label{which_warnings:results}
     \end{subfigure}
 \end{figure}


 Ruthruff et al. \cite{actionable_sa} use \textit{logistic regression} models to not only reduce the number of false positives in the output of SA tools, but also to predict actionable warnings. Warnings are not always acted on by developers even if they reveal true defects. The reason may be that the defects may have little impact and require significant effort for little perceived benefit. Furthermore, they introduce a statistical methodology for discarding features with low predictive power and thus avoiding the capture of expensive data. Information to build the models is mainly drawn from: (a) light-weight code complexity metrics for post-release bug prediction, (b) file (history) information to predict fault counts within individual files. The features include the history of warnings, source code characteristics, churn factors, and warnings descriptors (\cref{act:metrics}).

 Their screening methodology, for selecting an independent subset of predictor features, consists of up to four stages, and attempts to identify at least six predictive features. The stages respectively consider 5\%, 25\%, 50\%, 100\% of the warnings, continuously removing features with low predictive power. One of the reasons to consider this cost-effective approach is that it may be desirable to rebuild the models at different points in time, either because a significant number of new warnings have been reported, or the codebase has undergone substantial change.

 By considering a sample of around 1600 warnings (inspected by two engineers), and by using different models (with resulting different features) for classifying true positives and actionable warnings, they achieved an accuracy of 85\% for the former, and 70\% for the later (\cref{act:results}).

 \begin{figure}[H]
     \begin{subfigure}{.5\textwidth}
         \centering
         \includegraphics[scale=0.3]{./src/actionable_warnings_metrics.png}
         \caption{Some of the features considered for building the models}\label{act:metrics}
     \end{subfigure}%
     \begin{subfigure}{.5\textwidth}
         \centering
         \includegraphics[scale=0.3]{./src/actionable_warnings_results.png}
         \caption{Results for predicting true positives and actionable warnings}\label{act:results}
     \end{subfigure}
 \end{figure}


 Hanam et al. \cite{alert_patterns} present a method for differentiating actionable and unactionable alerts by finding alerts with similar code patterns (alerts with similar patterns are probably of the same type). They use a feature vector based on code characteristics at the site of each SA alert along with a classifier to build a model for predicting AA. They introduce the notion of \textit{alert patterns}, source code patterns employed by developers that are unactionable but are repeatedly flagged by SA tools (or similarly always actionable).

 To extract features from the site of the warning (and near it), lightweight program slicing is used. Backwards slicing is used to detect which statements could have affected the outcome of the seed statement (place of the alert). To speed up the slicing process, all external classes are excluded from the analysis and the depth is limited to the 5 nearest statements prior to the seed.

 By using the source code history of three projects to train and test their approach, they achieve considerably better results than the default ranking of a SA tool (57 vs 19 AA in the top 20\% of the alert list), and a slight improvement (6\%) than the existing techniques.

 \begin{figure}[H]
     \begin{subfigure}{.5\textwidth}
         \centering
         \includegraphics[scale=0.3]{./src/alert_patterns_slicing.png}
         \caption{Method for generating slices}\label{alert_patterns:slicing}
     \end{subfigure}%
     \begin{subfigure}{.5\textwidth}
         \centering
         \includegraphics[scale=0.3]{./src/alert_patterns_workflow.png}
         \caption{Workflow for classifying alerts}\label{alert_patterns:workflow}
     \end{subfigure}
 \end{figure}


 Venkatasubramanyam and Gupta \cite{incremental_sa} propose an incremental and lightweight approach to detect coding violations by using a learning system. They track warnings through the version history to detect patterns and determine which SA rules are important (and must be enabled). Their approach focuses on differential code analysis (filter/identify violations happening only on new parts of code), and on learning from the experts.

 Their methodology (\cref{incremental_sa:workflow}) consists of database that continuously stores information about SA rules. The initial version is build by mining (at least) the last three version of the software under analysis. To train the classifier (learning system) different features are used: patterns of code where SA violations are reported, impact of the violations on code quality, confidence level of the rules (probability that a rule gives a false positive), most commonly committed errors (reflecting the developers pattern of coding) and the most recently committed errors.

 New code changes made by developers are checked against the database. By using a patterns matching algorithm that compares new code with the patterns saved in the database, possible bugs can be detected. This approach potentially permits to run the SA tools less frequently, since code violations can be suggested by comparing against past saved patterns. They also suggest using \textit{first order logic} for capturing the context around rule violations and thus learning what factors produce a false/true positive.

 \begin{figure}[H]
     \centering
     \includegraphics[scale=0.3]{./src/incremental_sa_workflow.png}
     \caption{Workflow for incremental violation detection}\label{incremental_sa:workflow}
 \end{figure}


 Heckman and Williams \cite{model_building_actionable} present a generic approach for building actionable alert machine learning models and provide a comparative study of different algorithms tested of two Java systems. Their initial feature set consists of 51 alert characteristics originating from alert type and history, software metric, software history and source code churn.

 To collect data, they check out and build the program for each chosen revision (in practice they did that once in every 25 revisions) and collect alerts and their characteristics. Starting from the first revision, the sets of alerts between two revisions are compared, collecting information when alerts are opened and closed (and thus classifying them as actionable or unactionable).

 By trying different feature reduction strategies and different machine learning algorithms, they provide results for two systems. The number of selected alert characteristics ranged from 3/4 to 13/14 and both projects had 5 distinctive sets. That shows that the set of AC needs to be tailored for each project. The average metrics for all models (\cref{model_building:results}) show very good results. The difference between selected ACs and
 the best models between projects suggests that false positive mitigation models should be project-specific.

 \begin{figure}[H]
     \centering
     \includegraphics[scale=0.3]{./src/model_building_results.png}
     \caption{Average metrics for all models}\label{model_building:results}
 \end{figure}


 Liang et al. \cite{automatic_training_set} propose an automatic approach (\cref{automatic_training_set:workflow}) for constructing an effective training set for warning prioritization algorithms. They introduce the notion of "generic-bug-fix revisions" vs. "project-specific-bug revisions", which differentiate bug-fixing lines depending on the sort of bug that they deal with. SA tools are designed to catch generic bugs that are applicable to all projects, while most of the bugs are domain (project) specific. By restricting the training set to only those set of bugs that can be caught by the tools, models can be trained better and a higher accuracy can be reached.

 To identify generic-bug-fix revisions, they first limit the revision size to an empirically derived value (max 4 files changed). Then, they analyze the revision messages and using a natural language processing approach compare them against generic bug descriptions (by SA tools). If the similarity of these messages is above a certain threshold and the number of changed files is under the predefined limit, the revision is marked as a generic-bug-fix. To identify the generic-bug-related lines of a specific revision X, they start by analyzing all older revision than X, and backwardly calculate lines that were already present at X, when they were later changed by a generic-bug-fix revision.

 Using a \textit{K-Nearest Neighbor} classifier (trained on the first half of the revisions) paired with a feature selection algorithm, they achieve significantly better result than the tool output, especially in the first 20 warnings range. They found that using multiple SA tools gives a better result than single tool models (\cref{automatic_training_set:multiple_single}). The type training set has also an effect in the final results, models trained only with the project under analysis performed worse that models trained with extra projects (\cref{automatic_training_set:inter_intra}). That adding inter project data (at least in the case of open source projects) has a positive effect in the model predictions can be explained with the choice of focusing only on generic-bug-fix revisions.

 \begin{figure}[H]
     \begin{subfigure}{0.5\textwidth}
         \centering
         \includegraphics[scale=0.3]{./src/automatic_training_set_workflow.png}
         \caption{Workflow for constructing a training\\ set and a model}\label{automatic_training_set:workflow}
     \end{subfigure}%
     \begin{subfigure}{0.5\textwidth}
         \begin{subfigure}{.5\textwidth}
             \centering
             \includegraphics[scale=0.35]{./src/automatic_training_set_results1.png}
             \caption{Multiple vs. single tool results using training set}
             \label{automatic_training_set:multiple_single}
         \end{subfigure}\\
         \begin{subfigure}{.5\textwidth}
         \centering
         \includegraphics[scale=0.35]{./src/automatic_training_set_results2.png}
         \caption{Intra (A-a) vs inter (ABC-a) project training}
         \label{automatic_training_set:inter_intra}
         \end{subfigure}
     \end{subfigure}
 \end{figure}


 \subsubsection{Multiple Tools}
 \label{sec:multiple_tools}

 Flynn et al. \cite{multiple_classification} use \textit{Alert Fusion} (unifying alert information from different tools) and different classifiers to classify alerts as expected true positive (e-TP), expected true negative (e-TN) and indeterminate (I). The e-TP alerts are separately prioritized for code repair and the I-alerts are automatically ranked based on classifier confidence and a cost metric to fix the code flaw. 

 The authors used a total of 354 manually audited SA alerts, which there then mapped to standardized coding rule violations (CERT). Different types of classifiers were used, using different portions of data: trained to detect a single rule violation, trained for a single programming language, and all rule classifiers. The results vary from around 80 to 90\% accuracy, depending on the classifier type. The reliability of some of the results is doubtful since one of the major problems of the study was a lack of data.

 \begin{figure}[H]
     \begin{subfigure}{.5\textwidth}
         \centering
         \includegraphics[scale=0.3]{./src/multiple_classifiers_workflow.png}
         \caption{Workflow of building classifiers}\label{multiple_classifiers:workflow}
     \end{subfigure}%
     \begin{subfigure}{.5\textwidth}
         \centering
         \includegraphics[scale=0.3]{./src/multiple_classifiers_resulst.png}
         \caption{Results of all-rules classifiers}\label{multiple_classifiers:results}
     \end{subfigure}
 \end{figure}


 Ribeiro et al. \cite{multiple_ensemble} aim to reduce the false positive rate of an ensemble of static analyzers by using \textit{Decision Trees} and \textit{AdaBoost}. The goal is to make possible to combine the strengths of different analyzers without suffering too much from false positives. Their approach ignores source code characteristics, making it possible to be applied without any pre-processing step on the codebase.

 They use \textit{Juliet}, a synthetic C/C++ test suite which contains specific flaws with links to program code, to train and test the classifiers (see \cref{multiple_classifiers_ensemble:warnings} for the results of different tools on the set of selected test cases). The features to train the model include the tool name, number of warnings per file, warning category, number of neighboring warnings, number of warnings per file, and a boolean feature for each of the static analyzers.

 By combining weak decision tree classifiers with AdaBoost, they can reach a mean acccuracy of 80\% with a hundred trees, with precision and recall around 68\% and 96\% respectively. The ranking is done by sorting the warnings according to the probability assigned by the model, achieving a five time improvement over random ordering. The most important features in the classifier were the number of warnings per file and the tool name.

 \begin{figure}[H]
     \begin{subfigure}{.5\textwidth}
         \centering
         \includegraphics[scale=0.3]{./src/multiple_classifiers_ensemble_warnings.png}
         \caption{Labeled warnings per tool (from the extracted list of Juliet)}\label{multiple_classifiers_ensemble:warnings}
     \end{subfigure}%
     \begin{subfigure}{.5\textwidth}
         \centering
         \includegraphics[scale=0.3]{./src/multiple_classifiers_ensemble_results.png}
         \caption{Results of the classifier}\label{multiple_classifiers_ensemble:results}
     \end{subfigure}
 \end{figure}


 \subsubsection{Literature review papers}
 \label{sec:lit_review}

 Heckman and Williams \cite{literature_actionable} perform a systematic review of \textit{Actionable Alert Identification Techniques} (AAIT). The goal is to make an informed decision which AAIT to pair to an SA tool, in order to present relevant warnings to the tool users. An actionable alert is defined as an important, fixable anomaly. Different studies have estimated the amount of unactionable alerts ranging from 35\% to 91\%. The authors divide the tools into different categories, based on input type, approach used, and evaluation method.

 The categories of artifacts used by AAIT's are divided into five main categories: (a) alert characteristics (type, location), (b) code characteristics (metrics), (c) source code repository metrics (code churn), (d) bud database metrics, (e) dynamic analysis metrics (extracted during code execution). Most AAIT's combine more than one of these input categories.

 Approaches followed by AAIT's fall into seven main categories: (a) alert type selection (selecting altert types that are the most relevant for a codebase), (b) contextual information (limiting SA tools only to parts of code where that they can analyze well), (c) data fusion (combining multiple SA tools), (d) graph theory (system dependency graphs or repository history of changes), (e) machine learning, (f) mathematical and statistical models, (g) test case failures (generate test cases that demonstrate faults in the warning location). 

 Evaluation methodologies are divided into six categories: (a) baseline comparison (use a standard baseline), (b) benchmarks, (c) comparison to other AAIT's, (d) random and optimal order comparison, (e) train and test, (f) other.
 Classification AAIT's are evaluated using typical metrics as precision, recall, accuracy, and false positive rate, while Prioritization AAIT's are evaluated using correlation coefficients, statistical tests (chi-square), improvements over random, AUC etc..\\\\


 Muske and Serebrenik \cite{survey_approaches} perform a systematic review of SA alarm handling techniques (\cref{survey:papers}). They define \textit{handling of alarms}  as: (a) post-processing to reducing the manual inspection effort (using correlation, clustering, ranking...), and (b) supporting manual inspection of alarms.

 Seven categories for identifying alarms are defined (\cref{survey:categories}): (a) clustering, (b) ranking, (c) pruning, (d) false positive elimination, (e) combination with dynamic analysis, (f) simplifying inspections, (g) design of light-weight SA tools (\textit{LSATs}).

 In \textit{clustering}, alarms are partitioned into several groups based on similarity/correlation. There are two sub-categories: sound clustering, where there is a guarantee of certain dependencies among clustered alarms, and unsound clustering, where there are no guarantees on dependencies/relationships. 
 In \textit{ranking}, alarms are prioritized and those more likely to be errors are output at the top of the list. Different techniques can be used to support ranking, such as statistical models, history of alarm fixes, user feedback etc... 
 In \textit{pruning}, alarms are classified as actionable or non-actionable. Machine learning techniques can be used to classify the alarms (using patterns from surrounding code and syntactic/semantic differences), or alarm delta identification can be used identify the alarms that are newly generated (useful for legacy code). 
 In \textit{false positive elimination}, more precise techniques like model checking and symbolic execution are used to eliminate false positives. This approach is more precise and automatic but faces the issues of non scalability and poor performance.
 In \textit{combining dynamic and static analysis}, SA alarms are checked if they are true errors. SA has been combined with test-case generation or slicing to find errors or extract more precise information.
 In \textit{simplifying manual inspection}, approaches are used to help the user in alarm inspection by making inspections more automatic/systematic. Different techniques are used, from rule and checklist based approaches, to improved visualisation, to automatically deriving possible alarm causes.
 In \textit{designing LSATs}, light-weight, scalable and shallow analysis tools are built to avoid generation of a large number of alarms. However there are no guarantees that all defects of a type will be uncovered.

 \begin{figure}[H]
     \begin{subfigure}{1\textwidth}
         \centering
         \includegraphics[scale=0.4]{./src/survey_sa_papers.png}
         \caption{Number of relevant papers per year and category}\label{survey:papers}
     \end{subfigure}\\
     \begin{subfigure}{1\textwidth}
         \centering
         \includegraphics[scale=0.4]{./src/survey_sa_categories.png}
         \caption{Summary of the approaches}\label{survey:categories}
     \end{subfigure}
 \end{figure}


\subsubsection{Comparative studies}
\label{sec:comparative_studies}
 
Heckman and Williams \cite{comparative_heckman} perform a comparative study of six alert ranking techniques on the \textit{Faultbench} dataset: (a) Actionable Prioritization models that are based on the assumption that alerts sharing a type/location are like to be all actionable or non-actionable, (b) Alert Type Lifetime models that prioritize alert types by their average lifetime (important alerts are fixed quickly), (c) Check 'n Crash that automatically generates unit test cases and checks if the test fails (alert is then considered actionable), (d) History-Based Warning Prioritization models that uses commit messages and code changes in the source code repository to prioritize alert types, (e) Logistic Regression models that are trained on thirty-three alert characteristics and predict the probability of an alert being actionable, (f) Systematic Actionable Alert Identification that collects a number of alert characteristics and tries to find the best subset of these characteristics and the best machine learning models that optimizes accuracy and precision.

On each of the three test projects of the benchmark, there is a different winner (based on accuracy), with Systematic Actionable Alert Identification and Logistic Regression models that generally perform better. There is also a trend where precision and recall decrease with the amount of analyzed revisions (70, 80 or 90\%). That can be explained by the fact that the balance between actionable and unactionable alerts is heavily shifted to the later. This trend can also be explained by the fact that these techniques can be better at identifying unactionable alerts than actionable alerts.


Allier et al. \cite{compare_framework} perform a comparison of different ranking algorithms based on their effort metric: average number of alerts to inspect to find an actionable one. They also focus on two other research questions, whether its better to rank alerts individually or alert types, and if there is a performance difference between statistical ranking methods and ad-hoc ones. They test six raking approaches ( six Java and Smalltalk systems): Aware, FeedbackRank and Z-Ranking which mainly use alert type and location, RPM that uses logistic regression with thirty-three alert characteristics, AlertLifeTime that prioritizes alerts on type and lifetime, and EFindBugs which prioritizes alert types based on their defect likelihood.

They found out that Aware and FeedbackRank perform significantly better than the other ranking approaches. In addition individual alert raking algorithms performed better than those that rank alert types. Also they did not find a clear distinction in performance between statistical and ad-hoc approaches.


 \subsection{Bug Prediction}
 \label{sec:bug_prediction}

 Nagappan et al. \cite{mining_metrics} use code complexity metrics to predict the likelihood of \textit{post-release} defects for new entities (failures that occurred in the field six months after the release). Although, according to their findings, these metrics are correlated to failure prone entities, there is no universal set of metrics that produces the best results. As a consequence, principal component analysis is used to choose the optimal set of features for a particular project. Information from bug databases and historical data is used to select the appropriate metrics.

 By analyzing a set of five large scale projects, they discovered the following results: (a) for each project a set of metrics can be found that correlates with post-release defects, (b) there is no single et of metrics that fits all projects, (c) predictors build using \textit{PCA} are useful for building regression models that predict post-release defects, (d) predictors are only accurate when obtain from the same or similar projects.

 This approach can be generalized to predict arbitrary measures of quality, as long as we can extract the right information from the project's history. The general workflow is the following: decompose system in entities, build a function that assigns a quality measure to an entity, have a set of metrics and a metric functions that assigns a value to each entity, determine correlation of metrics to the quality measure and use PCA to select the most relevant set, use the principal components to predict quality of new entities.

 % \begin{figure}[H]
 %     \centering
 %     \includegraphics[scale=0.3]{./src/mining_metrics_overview.png}
 %     \caption{Some of the features considered for building the models}\label{mining:overview}
 % \end{figure}

 \begin{figure}[H]
     \begin{subfigure}{.5\textwidth}
         \centering
         \includegraphics[scale=0.3]{./src/mining_metrics_overview.png}
         \caption{Workflow for predicting defects on new entities}\label{mining:overview}
     \end{subfigure}%
     \begin{subfigure}{.5\textwidth}
         \centering
         \includegraphics[scale=0.2]{./src/mining_metrics_metrics.png}
         \caption{The set of complexity metrics considered}\label{mining:metrics}
     \end{subfigure}
 \end{figure}


 Giger et al. \cite{prediction_method} present bug prediction models at method level. In comparison to previous file or module level techniques, this increases the granularity of the prediction and thus reduces manual inspection (developers don't have to inspect a whole file). 

 The models are based on source code metrics that are applicable on method level (\cref{method_prediction:metrics}) while change metrics are based on fine-grained operations extracted from AST comparisons (tree edit operations needed to transform one AST into the other, combined with semantic information from the source code, \cref{method_prediction:ast}).

 By using an extensive test set of multiple open source Java projects and by labeling each method as bug-prone or not bug-prone (using historical VCS data) they were able to measure the efficacy of different classifiers. The classifiers were trained with both source and change metrics and each separately. The source metrics alone performed significantly worse than the other two and suffer from low precision values (around \%50). The change metrics (combined or not with the source ones) perform significantly better (> 80 \% precision) and the type of classifier does not significantly affect the results (\cref{method_prediction:results}).

 \begin{figure}[H]
     \begin{subfigure}{.5\textwidth}
         \centering
         \includegraphics[scale=0.3]{./src/method_prediction_metrics.png}
         \caption{Method level metrics used for prediction}\label{method_prediction:metrics}
     \end{subfigure}%
     \begin{subfigure}{.5\textwidth}
         \centering
         \includegraphics[scale=0.4]{./src/method_prediction_results.png}
         \caption{Precision, recall and AUC results for the metric sets}\label{method_prediction:results}
     \end{subfigure}\\
     \begin{subfigure}{\textwidth}
         \centering
         \includegraphics[scale=0.4]{./src/method_prediction_ast.png}
         \caption{Fine grained code changes extracted from AST comparisons}\label{method_prediction:ast}
     \end{subfigure}  
 \end{figure}


 Wang et al. \cite{predict_deeplearning} leverage deep learning to automatically learn semantic features from source code. The aim is to apply this knowledge into defect prediction, which traditionally uses syntactic features to build the models. In order to make accurate prediction, the features need to be discriminative, but traditional features cannot distinguish code regions with different semantics (see for example \cref{deeplearning:example}).

 A Deep Belief Neural Network is used to learn the semantic features from input vectors that contain tokens extracted from the AST's (code is parsed into tokens, tokens are then mapped into integers, which then form the vectors). Three main categories of AST nodes are extracted: a) nodes of method invocations and class instance creations, b) declaration nodes, c) and control flow nodes (see \cref{deeplearning:workflow} for the general workflow).

 To handle noise in data, the edit distance between the token sequences along with the \textit{Closest List Noise Identification} approach is used (compare instance label against its k-nearest neighbors). Additionally, infrequent tokens are filtered out of the training process.

 The DBN is tested against models trained with traditional features and models trained with the AST nodes. As can be seen from \cref{deeplearning:results}, the DL approach outperforms the traditional methods, with an average improvement in precision and recall of 14\% and 11\% respectively.

 \begin{figure}[H]
     \begin{subfigure}{1\textwidth}
         \centering
         \includegraphics[scale=0.5]{./src/deeplearning_example.png}
         \caption{Example of programs with same syntax (tokens) but different semantics}\label{deeplearning:example}
     \end{subfigure}\\
     \begin{subfigure}{\textwidth}
         \centering
             \makebox[\textwidth]{\includegraphics[scale=0.5]{./src/deeplearning_workflow.png}}%
         \caption{Workflow for the semantic learning process}\label{deeplearning:workflow}
     \end{subfigure}\\
     \begin{subfigure}{1\textwidth}
         \centering
         \includegraphics[scale=0.4]{./src/deeplearning_results.png}
         \caption{Precision, recall and F1 score for semantic vs syntactic features}\label{deeplearning:results}
     \end{subfigure}
 \end{figure}


 Yang et al. \cite{dl_jit_prediction} propose a deep learning technique to detect defect-prone changes (just-in-time defect prediction, i.e. inside commits). The advantage of this granularity is that there is a smaller amount of code to check and that it is easy to decide which developer should fix a bug (the one who committed the code). They use a two phase approach: a feature selection phase and a machine learning phase.

 The feature selection phase is to decide the best set of features to use to train the model. The data is pre-processed to in two steps: data is first normalized and then random under-sampling is used to balance the categories of buggy or not buggy changes. Since in logistic regression each feature is calculated independently, new features cannot be created by combining existing ones. For that reason, they leverage \textit{Deep Belief Networks} to generate a more expressive feature set. 

 The logistic regression model is trained with the new feature set and evaluated with a cost effectiveness measure defined as the percentage of bugs that can be discovered by inspecting the top (most relevant) 20\% lines of code. On average 50\% of bugs can be found in the top 20\% LOC, and the feature processing step helps logistic regression achieve better results than previous approaches (\cref{dl_jit:results}).

 \begin{figure}[H]
     \centering
     \includegraphics[scale=0.5]{./src/dl_jit_results.png}
     \caption{Improvements over classic logistic regression approach}\label{dl_jit:results}
 \end{figure}



 \subsection{Static Analysis tools in practice}
 \label{sec:tools_practice}


 Beller et al. \cite{analysis_sa_usage} conduct a large scale evaluation of how SA tools are used in practice in open-source systems. They study the prevalence of SA tools, their configurations and how they evolve.

 By performing a survey on a 36 open-source projects, they found out that most of them used SA tools, with a relevant subset using more that one (\cref{sa_analysis_survey:usage}). Although, most of them run the tools sporadically and without enforcing them.

 A configuration file of a SA tool, shows what rules developers deem important (enable), and what they do not deem important (disabled, perhaps because of a high false positives rate). The contents of a configuration file are hence an important indicator of how developers use SAs and how well the tool’s default settings reflect its use.

 In the analyzed projects, most enabled rules belong to the maintainability category, and only 35\% of the enabled rules belong to a functional category. Both the majority of actively enabled and disabled rules are maintainability-related.

 Most configurations change or reconfigure rules from the default configuration, but typically only one rule. Most changes are small, and a third of them happen in the first week of creation of the configuration. Also, most configuration files never change (\cref{sa_analysis_survey:changes}).

 \begin{figure}[H]
     \begin{subfigure}{1\textwidth}
         \centering
         \includegraphics[scale=0.4]{./src/sa_analysis_survey_usage.png}
         \caption{Survey results of 36 open-source systems using SA tools}\label{sa_analysis_survey:usage}
     \end{subfigure}\\
     \begin{subfigure}{1\textwidth}
         \centering
         \includegraphics[scale=0.4]{./src/sa_analysis_survey_changes.png}
         \caption{Changes in configuration of SA tools}\label{sa_analysis_survey:changes}
     \end{subfigure}
 \end{figure}


 Imtiaz et al. \cite{how_act_sa} analyze the SA usage of five large open-source systems (the tool is \textit{Coverity}). They study the amount of actionable alerts, time for fixing alerts and the size of fixes.

 They discover that 80\% of alerts belong to 20\% of the alert types and that the actionability rate varies from 27\% to 49\% depending on the project (\cref{how_use_sa:actionable}). Also in the case of actionable alerts, 20\% of the types causes 80\% of the actionable alerts.

 The median lifespan of actionable alerts varies between projects, ranging from 36 to 245 days, and the complexity of code changes is generally low. This means that developers generally take a long time to fix the alerts despite the fixes being low in complexity.

 To increase the developer interaction with SA tools they suggest two solutions: (a) prioritizing the most critical alerts and (b) providing an estimate for the fix effort.

 \begin{figure}[H]
     \begin{subfigure}{1\textwidth}
         \centering
         \includegraphics[scale=0.4]{./src/how_use_sa_actionable.png}
         \caption{Actionability results for total alerts}\label{how_use_sa:actionable}
     \end{subfigure}\\
     \begin{subfigure}{1\textwidth}
         \centering
         \includegraphics[scale=0.4]{./src/how_use_sa_topalerts.png}
         \caption{Top 10 alert occurrences for C/C++}\label{how_use_sa:top}
     \end{subfigure}
 \end{figure}


 Habib and Pradel \cite{how_many_bugs} study how many of all real-world bugs static bug detectors find. The results of their study show that: (a) static bug detectors find a non-negligible amount of all bugs, (b) different tools are mostly complementary to each other (see \cref{how_many_bugs:partition}), and (c) current bug detectors miss the large majority of the studied bugs.

 The three bug detectors together reveal 27 of the 594 studied bugs (4.5\%). Some of the missed bugs could have been found by variants of the existing detectors, while most of them are domain-specific problems that do not match any existing bug pattern that the SA tool have. By manually analyzing a small subset of 20 bugs, 14 of them were domain-specific and not related to any pattern supported by the checkers, while 6 of them were near-misses that could have been detected with a more powerful variant of the tool.

 They also found that the majority of bug fixes is limited in size and that most bugs are clustered on a small percentage of files (\cref{how_many_bugs:fixsize}).

 \begin{figure}[H]
     \begin{subfigure}{1\textwidth}
         \centering
         \includegraphics[scale=0.4]{./src/how_many_bugs_partition.png}
         \caption{Distribution of found bugs per tool}\label{how_many_bugs:partition}
     \end{subfigure}\\
     \begin{subfigure}{1\textwidth}
         \centering
         \includegraphics[scale=0.4]{./src/how_many_bugs_fixsize.png}
         \caption{Bugs/file and lines/bugfix}\label{how_many_bugs:fixsize}
     \end{subfigure}
 \end{figure}


 Sadowski et al. \cite{sa_google} provide an overview of the process that Google underwent to increase the developer interaction with SA tools. They list a number of shortcomings that hinder large scale adoption of such tools and suggest solutions that proved effective in their company. 

 The main reasons developers ignore of lose faith in SA tools are: (a) \textbf{no integration in workflow} (most important reason), (b) non actionable warnings, (c) reported bugs do not manifest in practice, (d) suggested bug is too risky/expensive to fix, (e) warnings are not understood.

 Google switched from the dashboard based \textit{FindBugs} tool (whose warnings were mostly ignored for two main reasons: developers lost faith because of false positives or alerts that were not important, and because the warnings came to late in the development workflow), to another better integrated approach.
 According to their findings, reporting issues sooner is better: moving as many checks into the compiler is the way to go. When possible, fixes are suggested or carried out automatically. A second place to show alerts that relate to high impact bugs, is the code review platform (for alerts with no simple fix). Code review is also a good context for reporting relatively less-important issues like stylistic problems or opportunities to simplify code.

 Another key point in making SA tools more valuable for the developers, is integrating their feedback, whether they accept or not the alerts proposed by the tool (for ex. adding a button for each alert \textit{Useful/Not Useful}).
 An additional workflow integration point is \textit{gating commits}: blocking a commit when a check fails (used for check with a low false positive rate).


% \subsection{Available tools}
% Open source tools found:
% \begin{itemize}
%     \item Infer
%     \item CppCheck
% \end{itemize}

% \subsection{Ideas?}
% \begin{itemize}
%     \item \textbf{Feasibility study}: combining different techniques (\cite{survey_approaches}). Pipelining or parallel?
%     \item mine bug messages / fixes to extract useful information?
%     \item use \cite{mining_metrics} to predict actionable warnings
%     \item lightweight local (file level?) execution likelihood
%     \item z-ranking or weighted warnings for cpp core guidelines
%     \item use ignore commands found in code for pruning future warning reports
%     \item see \cite{predict_deeplearning} introduction for a list of feature selection (also links to interesting papers)
%     \item To prune noisy data: mislabeling data detection approach Closest List Noise Identification (CLNI).
%     \item deep learning to classify warnings
%     \item single rule classifiers?
%     % \item consider the psycological aspect (coverity paper and \cite{sa_google})
%     \item when using multiple tools, consider the warning mapping (\cite{analysis_sa_usage}, \cite{multiple_ensemble}, \cite{multiple_classification}).
%     \item identify most relevant alert categories (80\% of actionable alerts come from 20\% of the types)
%     \item provide estimate for the fix?
%     \item less critical warnings as code review (+button, useful or not)
%     \item different checkers can be more useful for different teams
%     \item age of files is important!!! -> incremental detection
%     \item \textbf{mine patterns in ignored alerts?}
%     \item which context information to extract? first order logic approach \cite{incremental_sa} or sat tools?
%     \item can slicing be used?
%     \item differential changes as features?
%     \item cache locality?
%     \item check if pdg of clang can be useful
%     \item z ranking - core guidelines - based on file level (absence of warning -> successful check)
%     \item correlation tests between errors/warnings and exec likelihood/method bug score
%     \item use clang info on bug related lines
%     \item CLUSTER ALERTS!!!
%     \item \textbf{ignore closed alerts by clang-diagnostic-error}
%     \item fix number of authors
%     \item \textbf{consider the correlations between techniques}
%     \item use \textbf{clone information to cluster reports}
%     \item \textbf{CHECK WARNING AVERAGE LIFETIME: for remaining warnings unactionable if lifetime bigger than average}
% \end{itemize}


% \subsection{Acronyms}
% \begin{itemize}
%     \item SA - Static analysis
%     \item AA - actionable alert
%     \item AST - Abstract syntax tree
%     \item DL - Deep learning
% \end{itemize}


% !TEX root = main.tex

\section{Ranking the output of Static Analysis}\label{sec:ranking}
% Describing How You Solved the Problem or Answered the Question

% This part of the thesis is much more free-form. It may have one or several sections and subsections. But it all has only one purpose: to convince the examiners that you answered the question or solved the problem that you set for yourself in Section 4. So show what you did that is relevant to answering the question or solving the problem: if there were blind alleys and dead ends, do not include these, unless specifically relevant to the demonstration that you answered the thesis question. 

We analyze the performance of four different approaches to ranking SA alerts:
\begin{itemize}
	\item Using a Bayes Network for prioritizing alerts depending on location information.
	\item Predicting actionable alerts using ML algorithms based on different code/change metrics.
	\item Prioritizing alerts that point at bugs: analyze the code history to see which alerts pointed to fixed bugs and rank alert types based on that information.
	\item Detecting bug-prone methods and prioritizing alerts to those parts of code.
	%	\item Combining the three aforementioned methods, where each one focuses on a particular part of the dataset/alert types.
\end{itemize}

\subsection{Feedback Rank with Z-Score}

%\henrique{This section is named Feedback Rank and Z-Score, and the next Section is named Z-Score. If there is Z-score is not described in this section (but detailed in the next one) it should not be on the title.}
%\kleidi{Added two subsubsections, one for each.}
% \textbf{TODO: maybe try a different correlation (author/alert type/?)}
% check vs naive bayes - trade off with speed

\subsubsection{Feedback Rank}

Feedback Rank~\cite{correlation_exploitation} is a simple technique that ranks alerts on the probability of them being actionable.
It takes as input three location features for predicting if an alert is actionable: package, file, and function where the alert being analyzed is situated. 

%By constructing a Bayesian Network based on these three features it produces a probability ranking about which alerts are most useful. 
%\henrique{It is very difficult to understand what you mean in the last sentence. Only after reading the next sentence and looking at the Figure below that I got what you meant. You may want to re-write the paragraph as it is confusing}
%\henrique{You should detail the 'history of the project' which is probably one of the key elements to your ranking.}
%\kleidi{Removed the confusing sentences and explained better below.}

As explained on the literature review (\cref{lit:fbrank}), Feedback Rank is based on the assumption that true and false positives are clustered by code locality. The project space is divided into two major regions, one that contains mostly true positive, and one that contains mostly false positives (\textit{TPRegion, TFRegion}). Each package, file or function is considered to belong to one of these two regions. 

A Bayesian Network (BN) is used to calculate the probabilities of an alert or cluster of alerts belonging to a certain region. The network consist thus of one node for each of the artifacts (package, file, function), which in turn connect to a node representing alerts that belong to that specific location combination (see \cref{simple_bn}).  
The initial configuration of the network is learned from historical data (actionable alerts calculated as explained on \cref{data_collection}). The probabilities of each artifact belonging to one of these two regions, \textit{TPRegion} or \textit{TFRegion} are adjusted when training the network with the extracted data.

\begin{figure}[H]
	\centering
	\includegraphics[scale=0.2]{./src/bayes_example.png}
	\caption{Example of a simple Bayes Network for predicting 3 alerts}
	\label{simple_bn}
\end{figure}

According to the Kremenek et al. \cite{correlation_exploitation}, Feedback Rank is supposed to be an online ranking system: if we inspect a report and know its value, the probabilities of the parents are re-calculated. In this implementation, a static version is used. That means that the network is trained once and remains unchanged when predicting the rest of the alerts.

To construct the BN, we used the Pomegranate library \cite{pomegranate}, trained with the extracted alert data from the version history. 

\subsubsection{Z-Score}

%\henrique{Carefull on the overuse of passive voice. Every time you write "X is used" try to see if you can rewrite the sentence in active voice as "we use X".}
To break ties when the probabilities provided by the BN are equal between alerts, the Z-Score metric is used based on the number of alerts on the same file. Z-Score is used in Z-Ranking \cite{z-ranking}, which makes use of the observation that the most reliable error reports are those that generated few failed checks and many successful checks, since the actual amount of bugs in code is relatively small (see \cref{lit:zranking} for more details). 

The \textit{z-test} statistic, which measures how far an observed value is from the real population, in this case produces a large positive \textit{z-score} when there are few errors and many successes, and a large negative \textit{z-score} when there are few successes and many errors.

To make use of the Z-Score, an approximation is made. The granularity used for calculating the scores of alerts is based on file level (how many actionable/unactionable alerts of a certain type in a file), instead of the original granularity of the alert (for example an alert that only works on \textit{for loops}). This approximation is made because we do not know for each alert in which code construct it works on. Also, since it is only used as a tie-breaker, a high precision is not indispensable.

\subsection{Detecting Actionable Alerts via Machine Learning}
% classifiers used, how to train (grid search), overfitting, ways to measure

Different techniques focused on automatically classifying alerts in true/false positives or actionable/unactionable alerts, by constructing classifiers based on code or change metrics~\cite{actionable_sa, model_building_actionable}.

Instead of classifying alerts as true or false positives, we focus on Actionable Alerts which are alerts that are deemed important by the developers (not restricted to the type of alerts, but also to the context on which it manifests itself). Actionable alert is a less restrictive definition and makes it easier to collect data. Classifying alerts as true or false would require an oracle or a large and representative dataset generated manually. Moreover, from a developer's perspective, AAs can be more useful. A true positive alert may not be necessarily important to developers and thus be equally useless as a false one (e.g, low severity, no impact on the user side).

We followed the example of Heckman and Williams~\cite{model_building_actionable} to conduct our research since it contains an agglomeration of alert characteristics (AC) collected from other papers.

The workflow, as explained in \cref{data_collection}, consists of iterating through the version history, collecting alerts characteristics and keeping track which alerts disappear (considered as actionable). ACs are then later used as features in ML algorithms with actionability being the target to predict.

We use the scikit-learn library~\cite{scikit-learn} to perform ML experiments in our evaluation.

\subsubsection{Alert Characteristics}

%\henrique{Is it supposed to be: "Table X shows the collected features ..." ?}
The collected features can be classified in four main categories: alert information, source code metrics, churn metrics and version history information. The complete list of the collected features can be seen on \cref{act:featuresList}.

%
%\paragraph{Alert information:} (a) package name (or folder), (b) file name, (c) file extension, (d) alert type, (e) alert category (security, core guidelines...), (f) method signature.
%
%\paragraph{Source code metrics:} (a) number of statements, (b) number of methods, (c) number of classes, (d) cyclomatic complexity.
%
%\paragraph{Version history:} (a) alert open revision, (b) developers who made changes from the open revision of an alert to revision under analysis, (c) file creation revision, (d) file deletion revision, (e) latest modification revision.
%
%\paragraph{Churn metrics:} (a) added lines, (b) deleted lines, (c) growth (added-deleted), (d) total modified (added+deleted), (e) percent modified.
%
%\paragraph{Aggregate characteristics:} (a) total alerts for revision, (b) total open alerts for revision, (c) alert lifetime, (d) file age, (e) alerts for artifact (method, file, package), (f) staleness (amount of time since last change of file, method, package).


%\henrique{Where is this table number and title/caption. It is also important that the table is explicitly mentioned on the text.}

\begin{longtable}[c]{@{}ll@{}}
	\caption{List of features used to predict actionable alerts}
	\label{act:featuresList}\\
	\toprule
	\textbf{Feature}            & \textbf{Description}                                               \\* \midrule
	\endfirsthead
	%
	\endhead
	%
	\bottomrule
	\endfoot
	%
	\endlastfoot
	%
	\textit{category}           & Alert category                                                     \\
	\textit{type}               & Alert type                                                         \\
	\textit{function}           & Name of function where alert is located                            \\
	\textit{class}              & Name of class where alert is located                               \\
	\textit{file}               & Name of file where alert is located                                \\
	\textit{package}            & Name of folder where alert is located                              \\
	\textit{openRevision}       & Revision number when alert appeared (default base revision)        \\
	\textit{alertLifetime}      & current revision number - open revision number for alert           \\
	\textit{}                   &                                                                    \\
	\textit{functionSize}       & Number of statements in function where alert is located            \\
	\textit{fileSize}           & Number of statements in file where alert is located                \\
	\textit{nrFunctions}        & Number of functions in file                                        \\
	\textit{nrClasses}          & Number of classes in file                                          \\
	\textit{nrParameters}       & Number of parameters in function where alert is located            \\
	\textit{functionComplexity} & Cyclomatic completicy of function where alert is located           \\
	\textit{}                   &                                                                    \\
	\textit{addedFile}          & number of lines added to file in this revision                     \\
	\textit{deletedFile}        & number of lines deleted from file in this revision                 \\
	\textit{modifiedFile}       & addedFile + deletedFile                                            \\
	\textit{growthFile}         & addedFile - deletedFile                                            \\
	\textit{addedTotal}         & number of lines added in this revision                             \\
	\textit{deletedTotal}       & number of lines deleted in this revision                           \\
	\textit{modifiedTotal}      & addedTotal + deletedTotal                                          \\
	\textit{growthTotal}        & addedTotal - deletedTotal                                          \\
	\textit{}                   &                                                                    \\
	\textit{firstChangeFile}    & Revision number of first file change (default base revision)       \\
	\textit{lastChangeFile}     & Revision number of last file change (default base revision)        \\
	\textit{firstChangePackage} & Revision number of first change in package (default base revision) \\
	\textit{lastChangePackage}  & Revision number of last change in package (default base revision)  \\
	\textit{fileAge}            & lastChangeFile - firstChangeFile                                   \\
	\textit{fileStaleness}      & current revision - last change revision for file                   \\
	\textit{packageStaleness}   & current revision - last change revision for package                \\
	\textit{lastKnownDev}       & Last developer to change file                                      \\
	\textit{mostPresentDev}     & Developer that made most changes to file                           \\* \bottomrule
\end{longtable}

%\subsubsection{Pre-processing data}
%
%***mostly categorical data -> tree based algorithms\\
%\textbf{***do not use label encoding with nominal data (false order)}\\
%\textbf{***try other encoding techniques}
%
%\paragraph{Label encoding} Some algorithms need data in numerical form. Since most of the features are categorical data, we need a way to convert them to numerical. A simple approach was chosen, label encoding, which assigns an integer to each value of a category. Other approaches such as One-Hot encoding would increase the amount of data a lot (ex. consider one-hot encoding the file where an alert a method is from, which is a feature with high cardinality).
%
%\paragraph{Imbalanced dataset} A quick look at the number of closed or not closed alerts shows that the dataset is heavily unbalanced. In order to train the ML algorithms we need first to balance the dataset. Two main techniques were tried: random undersampling and SMOTE (by using the library from \cite{imblearn}).\\
%
%\textbf{TO DO:}\\
%-cite paper where oversampling works better\\
%-list other metrics\\
%-explain balancing techniques?
%
%\paragraph{Scaling} While for tree based algorithms scaling is not needed, for other algorithms like Logistic Regression, scaling the inputs can make a difference.
%
%\paragraph{Missing values} In some ACs data may be missing, for example...
%
%\paragraph{Feature selection} The features needed to train the models vary from project to project. Since collecting features that have little to no impact on algorithm performance is a waste of resources, techniques can be applied to select the most representative subset of features. Two main techniques were used: PCA and Recursive Feature Elimination.
%\textbf{TODO: COMPARISON FULL FEATURES VS REDUCED}
% correlations?

\subsection{Bug related lines}

Another automatic way to determine which alerts are useful is to check if they pointed to lines that were changed during bug fixes. By doing so, we are regarding as extra valuable those alerts that potentially signaled future bugs. The concept of bug related lines (BRL) is used by Kim and Ernst~\cite{which_warnings}, and by Liang et al.~\cite{automatic_training_set}.

BRLs are calculated as follows:
\begin{itemize}
    \item We start at a base version of the source code and iterate backwards to a target revision. 
    \item If revision under analysis is a bug-fixing revision (contains a bug ID) collect changed/deleted code lines from the version history.
    \item If those collected lines were present in the code at least since the target revision, we consider them bug related lines.
    \item Continue iterating backwards, collecting BRLs. If previous lines that were considered BRL were changed before reaching the target revision, we remove them from the set. 
\end{itemize}

Figure~\ref{fig:calculating-brls} shows a summarization of the general process we follow to calculate BRLs.

%\henrique{Every Figure, table, algorithm, chart, and diagram should be explicitly mentioned in the text. For example, you could put the following sentence below to mention the figure. }

\begin{figure}[H]
    \centering
    \includegraphics[scale=0.3]{./src/brl_example.png}
    \caption{Calculating BRLs (\cite{which_warnings})}
    \label{fig:calculating-brls} %% *** Put labels in the figures, tables, etc, to make it easier to reference them.
\end{figure}

% generic BRL vs project specific BRL

Since the number of collected lines and warnings can be rather limited, we extend the definition to \textit{Bug Related Methods} (BRM). Namely, we trace in which methods the BRLs belong to and also consider alerts inside those methods as valuable. That allows us to extend the dataset, but also potentially weakens the data by introducing more noise.

Following the approach used by Kim and Ernst~\cite{which_warnings}, we can calculate weights for warning types, and use the weights to rank alerts based on how good is that particular alert type in predicting bugs. The algorithm used by the authors is rather simple and calculates weights based on two input parameters, $\alpha$ and $\beta$. The former used to increase the weights of alert types belonging to the collected \textit{BRL/BRM} set, while the later used for the actionable alert set (\cref{weight_calc_algo}).

%Given that the number of alerts collected this way, is a lot less than the total number of alerts (heavy imbalance), this method is more suited for ranking alert types rather than individual alerts (as used in \cite{which_warnings}).
%\henrique{I did not understand what is the relation that makes it better suited for ranking alert types.}
%\kleidi{The amount of collected alerts is very small to use the same approach as the previous section. I removed the sentence.}

\subsection{Method Bug prediction}

Following the example of Boogerd and Moonen~\cite{static_profiling}, which aims to prioritize alerts based on the execution likelihood of the code pointed by the alert, we present a similar approach. Alerts pointing at potential bugs are the ones that can be considered the most important. The cost of detecting and fixing a bug is much lower if detected early in the development cycle. If we can predict which components will be likely to cause failures in the future, we can prioritize alerts that point to those components.

The appropriate granularity to use for bug prediction is at method level. File level granularity is too broad since a lot of files contain many lines of code, which would result in prioritizing a lot of alerts. On the other hand, finer granularity than method level would be too hard to predict.

We follow the example of Giger et al.~\cite{prediction_method} method. By using a combination of source code and change metrics and by exploiting the version history of the codebase, classifiers can be built that predict bug prone methods. The complete list of the collected features can be seen on \cref{bugpredict:features}.

\begin{figure}[H]
	\centering
	\begin{minipage}{.5\linewidth}
		\begin{algorithm}[H]
			\SetAlgoLined
			\KwData{Collected bug-related alerts and actionable alerts}
			\KwResult{Weighted alert types}
			$\alpha, \beta = x, y$\;
			$w_t = 0 \ for \ t \ in \ alertTypes$\;
			\For{alert in collectedAlerts}{
				$w_t = typeOf(alert)$\;
				\If{alert pointed to a BRL or BRM}{
					$w_t = w_t + \alpha$
				}
				\ElseIf{alert is actionable}{
					$w_t = w_t + \beta$
				}
			}
			$w_t = \frac{w_t}{|alerts \, of \, type \, t|}$
			\caption{Alert type priotitization algorithm}
		\end{algorithm}
	\end{minipage}
	\label{weight_calc_algo}
\end{figure}

\begin{longtable}[c]{@{}ll@{}}
	\caption{List of features used to predict buggy methods}
	\label{bugpredict:features}\\
	\toprule
	\textbf{Metric Name}          & \textbf{Description}                                                                                                                   \\* \midrule
	\endfirsthead
	%
	\endhead
	%
	\bottomrule
	\endfoot
	%
	\endlastfoot
	%
	\textit{methodHistories}      & Number of times a method was changed                                                                                                   \\
	\textit{authors}              & Number of distinct authors that changed a method                                                                                       \\
	\textit{stmtAdded}            & Sum of all source code statements added to a method                                                                                    \\
	\textit{maxStmtAdded}         & \begin{tabular}[c]{@{}l@{}}Maximum number of source code statements added \\ to a method body for all method histories\end{tabular}    \\
	\textit{avgStmtAdded}         & \begin{tabular}[c]{@{}l@{}}Average number of source code statements added\\ to a method body per method history\end{tabular}           \\
	\textit{stmtDeleted}          & \begin{tabular}[c]{@{}l@{}}Sum of all source code statements deleted from\\ a method body over all method histories\end{tabular}       \\
	\textit{maxStmtDeleted}       & \begin{tabular}[c]{@{}l@{}}Maximum number of source code statements\\ deleted from a method body for all method histories\end{tabular} \\
	\textit{avgStmtDeleted}       & \begin{tabular}[c]{@{}l@{}}Average number of source code statements\\ deleted from a method body per method history\end{tabular}       \\
	\textit{churn}                & \begin{tabular}[c]{@{}l@{}}Sum of stmtAdded - stmtDeleted over all\\ method histories\end{tabular}                                     \\
	\textit{maxChurn}             & Maximum churn for all method histories                                                                                                 \\
	\textit{avgChurn}             & Average churn per method history                                                                                                       \\
	\textit{decl}                 & \begin{tabular}[c]{@{}l@{}}Number of method declaration changes over all\\ method histories\end{tabular}                               \\
	\textit{cond}                 & \begin{tabular}[c]{@{}l@{}}Number of condition expression changes in a\\ method body over all revisions\end{tabular}                   \\
	\textit{elseAdded}            & \begin{tabular}[c]{@{}l@{}}Number of added else-parts in a method body\\ over all revisions\end{tabular}                               \\
	\textit{elseDeleted}          & \begin{tabular}[c]{@{}l@{}}Number of deleted else-parts from a method\\ body over all revisions\end{tabular}                           \\
	\textit{cyclomaticComplexity} & Current cyclomatic complexity of method                                                                                                \\
	\textit{nestingDepth}         & Current nesting depth of method                                                                                                        \\
	\textit{totalStatements}      & Current number of statements in method                                                                                                 \\
	\textit{nrPaths}              & Current number of paths in method                                                                                                      \\
	\textit{nrDeclarations}       & Current number of declarations in method                                                                                               \\* \bottomrule
\end{longtable}

% rank alerts in problematic zones higher - combine with BRL?

%\subsection{Combining the techniques}
%% \subsection{Individual alert tuning}
%% zranking cppcore guidelines
%% COMBINE DIFFERENT TECHNIQUES FOR DIFFERENT ALERTS
%
%\subsubsection{How to combine the strengths?}
%% deep learning? make it flexible, learn with time\\
%apply each technique to a subset of the warnings (which is more appropriate)
%
%\subsubsection{Better results?}
%Does it provide better results?



% !TEX root = main.tex

\section{Threats to Validity}\label{sec:threats}
%\henrique{In any empirical research, we must analysed and categorize the threats to validity of the experiments. The threats are classified into four cateegories: Construct Validity, Conclusion Validity, Internal Validity, and External Validity}

\henrique{I am impressed by the quality of the section. Nice work.}

\subsection{Construct validity}
%Construct validity focus on the relation between the theory
%behind the experiment and the observation(s). Even if we
%have established that there is a casual relationship between
%the treatment of our experiment and the observed outcome,
%the treatment might not correspond to the cause we think we
%have controlled and altered. Similarly, the observed outcome
%might not correspond to the effect we think we are measuring.

As already mentioned in \cref{data_assumptions}, the automatic nature of data collection poses a threat to validity. The collected dataset may contain an inflated amount of positive examples (either actionable alerts, bug related lines, or buggy methods). That is something that we cannot fully mitigate, but still, try to limit the risks by applying data cleaning approaches where applicable.

\subsection{Conclusion validity}
%Conclusion validity focus on how sure we can be that the
%treatment we used in an experiment really is related to the
%actual outcome we observed. Typically this concerns if there
%is a statistically significant effect on the outcome.

By using a mix of different metrics, that relate to classification and ranking, we can confidently make claims on the effect of ranking methods. 
First, we use metrics suitable for unbalanced data, so that the results of the majority class (unactionable alerts) does not overwhelm the results and mask classification problems for the minority class (actionable alerts). Second, we compare the ranked output of our methods with a random ranking to see if there is an improvement. That is specifically done on a balanced dataset, so that the random ranking can be fairly evaluated against the improved one (in contrast to a random ranking that only has few positive samples and will almost always perform worse than any other minor ranking improvement). Third, we train the algorithms on a balanced dataset, as to avoid potential bias.


\subsection{Internal validity}
%Internal validity focus on how sure we can be that the treatment
%actually caused the outcome. There can be other factors that
%have caused the outcome, factors that we do not have control
%over or have not measured.

By following the suggestion of Pascarella et al. \cite{performance_method_bug}, we use a horizontal train/test strategy, meaning that the train and test sets are extracted by sequentially dividing the sorted dataset (in order of alert appearance in time) in two parts. That way we avoid using dependent variables that are not available at prediction time on a real world scenario.


\subsection{External validity}
The scope of this thesis, conducted as an internship in a company, is to research the effectiveness of alert ranking techniques in a particular industrial environment. We do not test the generalizability of our results outside the context on which the thesis was conducted, and as a consequence, we make no claims on the behavior or performance in an external context.

% !TEX root = main.tex

\section{Conclusions}\label{sec:conclusion}
% You generally cover three things in the Conclusions section, and each of these usually merits a separate subsection:

% 1. Conclusions
% 2. Summary of Contributions
% 3. Future Research

% Conclusions are not a rambling summary of the thesis: they are short, concise statements of the inferences that you have made because of your work. It helps to organize these as short numbered paragraphs, ordered from most to least important. All conclusions should be directly related to the research question stated in Section 4. Examples:

% 1. The problem stated in Section 4 has been solved: as shown in Sections ? to ??, an algorithm capable of handling large-scale Zylon problems in reasonable time has been developed.
% 2. The principal mechanism needed in the improved Zylon algorithm is the Grooty mechanism.
% 3. Etc.

% The Summary of Contributions will be much sought and carefully read by the examiners. Here you list the contributions of new knowledge that your thesis makes. Of course, the thesis itself must substantiate any claims made here. There is often some overlap with the Conclusions, but that's okay. Concise numbered paragraphs are again best. Organize from most to least important. Examples:

% 1. Developed a much quicker algorithm for large-scale Zylon problems.
% 2. Demonstrated the first use of the Grooty mechanism for Zylon calculations.
% 3. Etc.

% The Future Research subsection is included so that researchers picking up this work in future have the benefit of the ideas that you generated while you were working on the project. Again, concise numbered paragraphs are usually best. 


\subsection{Conclusions}
\textbf{TO DO}\\
-got these results, because...\\
-importance on preprocessing\\
-initial results may be slightly better that normal, but the nature of data collection is limiting, need to be used in continuance\\ 

% \begin{itemize}
%     \item Ranking can be useful but only with enough data
%     \item X method performs better / is more useful than Y
%     \item Combining techniques give these different results
%     \item Context of industrial code
% \end{itemize}


\subsection{Summary of Contributions}
 
This thesis provides the following contributions:
\begin{itemize}
\item Building a workflow to extract information by making use of the Clang toolset and version history (SVN). 
\henrique{A contribution is something that was finished, that it is done. Anything starting with 'building' (or any other 'ing') is not a contribution because you are still doing it (as it is not finished). Even if I rewrite the sentence, I am not convinced that the workflow is a contribution.}
\kleidi{The thesis is made in the context of an internship, where in my opinion the focus is higher on implementation than a normal research. Since a considerable amount of time was put into extraction of the data (more difficult in a restricted company environment than a typical research), it is also meant as a justification that an adequate amount of time was spent in the making of this thesis. I do agree that in the research point of view it is not something considered valuable. I do not mind if it is removed, but don't want to give the wrong impression that the thesis might be something hacked together in little time.}
\henrique{One contribution will not be the sole cause that gives the impression of 'thesis hacked together in little time'. It is overacting to say so. I understand that you may be stressed out due to the deadline, but try to keep a cool head. Moreover, it does not matter if you chose to focus on implementation or research, a contribution in the thesis will be the same nonetheless. On that note, not everything you worked on is a contribution. I do not deny that you put a lot of work on extracting data. Now saying that is a contribution is a completely different thing. Workflows (most often) are not contributions. The process or method on a workflow could be a contribution, but I do not think yours is. The extracted data could be a contribution if it was a public dataset useful for other researchers to conduct comparative experiments, and it is not the case for your data either. Therefore, I do not see anything here that can be classified as a contribution. In the end, this is YOUR thesis, if you want to leave this as a contribution go ahead. You are more than free to follow my suggestions or not.}

\item The evaluation of SA ranking techniques on an industrial codebase. \henrique{I think you are missing a little more detail to make this into a proper contribution.}
\item A pre-processing procedure to deal with imbalanced and noisy data.
\item We conducted a comparative study of different approaches on a common codebase.
\item We demonstrate the utility of combining different ranking methods into an ensemble. \henrique{by using a Bayesian network?}
\kleidi{this is unfortunately a to do for the moment, hope to finish it tonight/tomorrow}
\henrique{I was wondering if the ensemble was performed by using a Bayesian network. If it is then it should be detailed in the contribution. }
\kleidi{that is what is holding me back for the moment, how to precisely combine them}

\end{itemize}

% -building workflow to extract information\\
% -evaluation on industrial code\\
% -comparison of utility between tools\\
% -evaluation of an ensemble technique

\subsection{Future Research}

%Analyze the parts where the prediction fails?

Future research can be focused on different aspects, the most important being reliable data collection. A classifier is as good as the data it was trained on, so new ways to collect actionable alerts in a more precise way are crucial to achieving better performance. Information Retrieval or Natural Language Processing techniques can be applied for example on bug descriptions to have a clearer connection between a bug and an alert.

Given also the limited amount of data, new or improved approaches that can generalize easily are needed. In that regard, research can focus on the type of features extracted from the code or version history that are discriminative enough to make correct classification even with limited data. The impact of pre-processing can be explored even further, an example being on how to deal with high-cardinality data better, best encoding methods, etc.

Furthermore, given the diverse nature of alerts current tools have, from finding bugprone construct, stylistic alerts, library-oriented alerts, to security or performance-oriented checks, different methods can be tailored that maximize performance within these subsets of alerts. For example, a simple method like Z-Ranking (\cite{z-ranking}) can be more suited to predict stylistic alerts than others.

In addition, a way to continuously improve the ranking algorithms needs to be put in place. Even though the initial performance may not be spectacular, if new resolved alerts are tracked consistently, performance will also rise accordingly. A rather naive implementation is to give warnings an identifier and include them in the commit messages if they were useful.

% -data collection assumptions, make it more reliable?\\
% -better track resolved alerts (google tricorder example)
% -information retrieval on bug messages
% TO DO



\nocite{*}
\bibliographystyle{unsrt}
\bibliography{references}

\end{document}